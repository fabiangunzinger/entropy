% !TEX root = ../entropy.tex

\section{Spending profiles predict emergency savings}%
\label{sec:results}

Table~\ref{tab:reg_has_inflows_main} shows the effect of entropy on the
probability of having emergency savings. Columns (1)-(3) show results for
unsmoothed entropy based on 9 categories, 48 categories, and merchant names,
respectively. Columns (4)-(6) results for smoothed entropy based on the same
variables. All models include user and year-month fixed effects, and standard
errors are clustered at the user-level. Confidence intervals are shown in
brakets.

\input{\tabdir/reg_has_inflows_main.tex}

Results for unsmoothed entropy suggest that a one unit increase in entropy is
associated with an increase in the probability of a user making at least one
transfer into their savings accounts of between 1.5 and 2.7 percentage points
-- an effect up to two times larger than that of a \pounds1000 increase in
monthly income.

Conversely, the effect for unsmooth entropy is smaller in magnitude but runs in
the reverse direction: a one-unit increase in the smoothed entropy score is
associated with a reduction in the probability of transferring money into
savings account of between 0.4 and 1.6 percentage points -- an effect that, in
absolute magnitude, is about equal to that of a \pounds1000 increase in monthly
income.

Overall, then, the effect of entropy in spending profiles is statistically and
economically significant, and robust across different definitions. In other
words, the scores seem to pick up a feature of the spending distribution that
is predictive of savings behaviour.

But how can we account for the opposite sign for smoothed and unsmoothed
entropy scores? We don't have the answer to this yet... 




