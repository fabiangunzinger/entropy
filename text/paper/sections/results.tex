% !TEX root = ../entropy.tex

\section{Spending profiles predict emergency savings}%
\label{sec:results}

Table~\ref{tab:reg_has_inflows_main} shows the effect of entropy on the
probability of building emergency savings in a given month. Columns (1)-(3)
show results for unsmoothed entropy based on 9 categories, 48 categories, and
merchant names, respectively. Columns (4)-(6) results for smoothed entropy
based on the same variables. All models include user and year-month fixed
effects, and standard errors are clustered at the user-level. 95\% confidence
intervals are shown in brakets.

\input{\tabdir/reg_has_inflows_main.tex}
\input{\tabdir/reg_has_investments_main.tex}

Results for unsmoothed entropy suggest that a one unit increase in entropy is
associated with an increase in the probability of a user making at least one
transfer into their savings accounts of between 1.5 and 2.7 percentage points
-- an effect up to two times larger than that of a \pounds1000 increase in
monthly income.

Conversely, the effect for unsmooth entropy is smaller in magnitude but runs in
the reverse direction: a one-unit increase in the smoothed entropy score is
associated with a reduction in the probability of transferring money into
savings account of between 0.4 and 1.6 percentage points -- an effect that, in
absolute magnitude, is about equal to that of a \pounds1000 increase in monthly
income.

Overall, then, the effect of entropy in spending profiles is statistically and
economically significant, and robust across different definitions. In other
words, the scores seem to pick up a feature of the spending distribution that
is predictive of savings behaviour.

Two questions remain: first, how can we understand the aspect of users'
spending profile that entropy captures and that is predictive of savings
behaviour? And second, why does smoothing entropy scores flip the direction of
the effect? We will address these in turn.



\begin{table}[htbp]
   \centering
   \tiny
   \begin{threeparttable}[b]
      \caption{\label{tab:reg_has_inflows_comp} Controlling for entropy components}
      \begin{tabular}{lcccc}
         \tabularnewline \midrule \midrule
         Model:                    & (1)             & (2)             & (3)              & (4)\\  
         \midrule
         \emph{Variables}\\
         Entropy (48 cats)         & 0.029$^{***}$   & 0.013$^{***}$   &                  &   \\   
                                   & [0.025; 0.033]  & [0.006; 0.021]  &                  &   \\   
         Entropy (48 cats, smooth) &                 &                 & -0.023$^{***}$   & -0.028$^{***}$\\   
                                   &                 &                 & [-0.025; -0.020] & [-0.034; -0.022]\\   
         Unique categories         &                 & 0.004$^{***}$   &                  & 0.004$^{***}$\\   
                                   &                 & [0.003; 0.005]  &                  & [0.004; 0.005]\\   
         Category counts std.      &                 & 0.002           &                  & -0.015$^{***}$\\   
                                   &                 & [-0.001; 0.006] &                  & [-0.018; -0.011]\\   
         Number of spend txns      &                 & 0.000$^{***}$   &                  & 0.001$^{***}$\\   
                                   &                 & [0.000; 0.001]  &                  & [0.001; 0.001]\\   
         Month spend               & 0.009$^{***}$   & 0.005$^{***}$   & 0.008$^{***}$    & 0.005$^{***}$\\   
                                   & [0.008; 0.009]  & [0.004; 0.006]  & [0.007; 0.009]   & [0.005; 0.006]\\   
         Month income              & 0.012$^{***}$   & 0.011$^{***}$   & 0.011$^{***}$    & 0.011$^{***}$\\   
                                   & [0.011; 0.013]  & [0.010; 0.012]  & [0.011; 0.012]   & [0.010; 0.012]\\   
         Has income in month       & 0.084$^{***}$   & 0.079$^{***}$   & 0.085$^{***}$    & 0.078$^{***}$\\   
                                   & [0.075; 0.092]  & [0.070; 0.087]  & [0.076; 0.093]   & [0.070; 0.087]\\   
         Income variability        & 0.001$^{*}$     & 0.000           & 0.001$^{*}$      & 0.000\\   
                                   & [-0.000; 0.001] & [-0.000; 0.001] & [-0.000; 0.001]  & [-0.000; 0.001]\\   
         \midrule
         \emph{Fixed-effects}\\
         User                      & Yes             & Yes             & Yes              & Yes\\  
         Year-month                & Yes             & Yes             & Yes              & Yes\\  
         \midrule
         \emph{Fit statistics}\\
         Observations              & 1,043,727       & 1,043,727       & 1,043,727        & 1,043,727\\  
         R$^2$                     & 0.45395         & 0.45498         & 0.45415          & 0.45515\\  
         Within R$^2$              & 0.00768         & 0.00956         & 0.00805          & 0.00986\\  
         \midrule \midrule
         \multicolumn{5}{l}{\emph{Clustered (User) co-variance matrix, 95\% confidence intervals in brackets}}\\
         \multicolumn{5}{l}{\emph{Signif. Codes: ***: 0.01, **: 0.05, *: 0.1}}\\
      \end{tabular}
   \end{threeparttable}
\end{table}




Columns (1) and (3) in Table~\ref{tab:reg_has_inflows_comp} replicate the
results for the 48-category-based unsmoothed and smoothed entropy measures
presented in Table~\ref{tab:reg_has_inflows_main} for reference. In columns (2)
and (4) we additionally control for the three entropy components discussed
above: the number of unique spend categories, the standard deviation of their
counts, and the total number of spending transactions. Including these
components has some effect: for unsmoothed entropy the magnitude of the
coefficient is less than half its size and the confidence interval is about
twice as wide, while for smoothed entropy the magnitude of the coefficient is a
little higher while the width of the confidence interval also roughly doubles.
However, both coefficients remain statistically significant and their
confidence intervals cover values that are also economically significant.
Hence, the results make clear that the results in
Table~\ref{tab:reg_has_inflows_main} cannot be attributed simply to the effect
of one or more of entropy's simple components.

Another way to think of entropy is that, as discussed in the introduction, it
might be a function of life circumstances that simultaneously determine
spending and savings behaviour.


The economic literature on scarcity documents that our minds tend to focus on
what is scarce and neglect what is not, concentrating our mental resources
where they are most needed but impeding decision-making in other domains,
possibly leading to impaired and short-sighted decision
making\citep{shah2012some, mullainathan2013scarcity, haushofer2014psychology}.
For instance, \citet{mani2013poverty} find that low-income shoppers in New
Jersey perform worse on cognitive tasks when first promoted to think about
their financial situation while the same prompts had no effect for wealthier
shoppers, and sugarcane farmers in India perform worse on similar cognitive
tasks shortly before the annual harvest (when money is scarce) than shortly
thereafter (when money is plentiful). A prediction of this theory is that
people with low levels of savings find it hard to make sound spending decisions
as thir account balance dwindles, because their minds are increasingly captured
by financial worries. If high entropy spending behaviour captures life stress,
we would thus think that it is individuals with lowerer levels of incomes, who
have less mental bandwith to deal with said stress, that are most effected.

The results below show results similar to those presented in
Table~\ref{tab:reg_has_inflows_main} above, but separately for unsmoothed and
smoothed income and separately for each income quintile and income variation
quintile, respectively.


\begin{table}[htbp]
   \centering
   \tiny
   \begin{threeparttable}[b]
      \caption{\label{tab:reg_has_inflows_entropy_tag_spend_z_inc_quint} Effect of entropy on P(has savings) by income quintile}
      \begin{tabular}{lccccc}
         \tabularnewline \midrule \midrule
         Model:             & (1)             & (2)             & (3)            & (4)            & (5)\\  
         Income quintile:   & 1               & 2               & 3              & 4              & 5 \\   
         \midrule
         \emph{Variables}\\
         Entropy (48 cats)  & 0.033$^{***}$   & 0.019$^{***}$   & 0.017$^{***}$  & 0.016$^{***}$  & 0.018$^{***}$\\   
                            & [0.028; 0.039]  & [0.013; 0.025]  & [0.011; 0.023] & [0.010; 0.022] & [0.012; 0.025]\\   
         Month spend        & 0.010$^{***}$   & 0.007$^{***}$   & 0.008$^{***}$  & 0.007$^{***}$  & 0.006$^{***}$\\   
                            & [0.009; 0.011]  & [0.006; 0.008]  & [0.006; 0.009] & [0.006; 0.008] & [0.005; 0.007]\\   
         Month income       & 0.015$^{***}$   & 0.010$^{***}$   & 0.012$^{***}$  & 0.013$^{***}$  & 0.008$^{***}$\\   
                            & [0.012; 0.019]  & [0.004; 0.017]  & [0.005; 0.019] & [0.008; 0.019] & [0.007; 0.010]\\   
         Income variability & -0.000          & 0.001           & 0.002$^{***}$  & 0.001$^{**}$   & -0.000\\   
                            & [-0.001; 0.001] & [-0.000; 0.002] & [0.001; 0.004] & [0.000; 0.002] & [-0.001; 0.001]\\   
         \midrule
         \emph{Fixed-effects}\\
         User               & Yes             & Yes             & Yes            & Yes            & Yes\\  
         Year-month         & Yes             & Yes             & Yes            & Yes            & Yes\\  
         \midrule
         \emph{Fit statistics}\\
         Observations       & 223,462         & 215,151         & 208,417        & 201,319        & 195,378\\  
         R$^2$              & 0.59138         & 0.58949         & 0.58354        & 0.58212        & 0.57162\\  
         Within R$^2$       & 0.00535         & 0.00180         & 0.00211        & 0.00191        & 0.00271\\  
         \midrule \midrule
         \multicolumn{6}{l}{\emph{Clustered (User) co-variance matrix, 95\% confidence intervals in brackets}}\\
         \multicolumn{6}{l}{\emph{Signif. Codes: ***: 0.01, **: 0.05, *: 0.1}}\\
      \end{tabular}
   \end{threeparttable}
\end{table}



\input{\tabdir/reg_has_inflows_entropy_tag_spend_sz_inc_quint.tex}


\begin{table}[htbp]
   \centering
   \tiny
   \begin{threeparttable}[b]
      \caption{\label{tab:reg_has_inflows_entropy_tag_spend_z_inc_var_quint} Effect of entropy on P(has savings) by income variability quintile}
      \begin{tabular}{lccccc}
         \tabularnewline \midrule \midrule
         Model:                       & (1)             & (2)            & (3)             & (4)             & (5)\\  
         Income variability quintile: & 1               & 2              & 3               & 4               & 5 \\   
         \midrule
         \emph{Variables}\\
         Entropy (48 cats)            & 0.020$^{***}$   & 0.014$^{***}$  & 0.022$^{***}$   & 0.028$^{***}$   & 0.021$^{***}$\\   
                                      & [0.014; 0.026]  & [0.008; 0.020] & [0.016; 0.028]  & [0.022; 0.034]  & [0.015; 0.027]\\   
         Month spend                  & 0.008$^{***}$   & 0.008$^{***}$  & 0.007$^{***}$   & 0.007$^{***}$   & 0.007$^{***}$\\   
                                      & [0.007; 0.009]  & [0.007; 0.009] & [0.006; 0.008]  & [0.006; 0.008]  & [0.006; 0.008]\\   
         Month income                 & 0.014$^{***}$   & 0.010$^{***}$  & 0.011$^{***}$   & 0.010$^{***}$   & 0.010$^{***}$\\   
                                      & [0.011; 0.017]  & [0.008; 0.012] & [0.010; 0.013]  & [0.009; 0.012]  & [0.009; 0.011]\\   
         Has income in month          & 0.092$^{***}$   & 0.076$^{***}$  & 0.064$^{***}$   & 0.055$^{***}$   & 0.067$^{***}$\\   
                                      & [0.073; 0.111]  & [0.056; 0.095] & [0.047; 0.081]  & [0.040; 0.070]  & [0.053; 0.081]\\   
         Income variability           & -0.001          & 0.006$^{**}$   & -0.000          & 0.000           & -0.004$^{**}$\\   
                                      & [-0.003; 0.001] & [0.001; 0.011] & [-0.004; 0.003] & [-0.001; 0.002] & [-0.007; -0.001]\\   
         \midrule
         \emph{Fixed-effects}\\
         User                         & Yes             & Yes            & Yes             & Yes             & Yes\\  
         Year-month                   & Yes             & Yes            & Yes             & Yes             & Yes\\  
         \midrule
         \emph{Fit statistics}\\
         Observations                 & 223,462         & 215,151        & 208,417         & 201,319         & 195,378\\  
         R$^2$                        & 0.61166         & 0.59951        & 0.59062         & 0.59379         & 0.63519\\  
         Within R$^2$                 & 0.00558         & 0.00398        & 0.00484         & 0.00579         & 0.00781\\  
         \midrule \midrule
         \multicolumn{6}{l}{\emph{Clustered (User) co-variance matrix, 95\% confidence intervals in brackets}}\\
         \multicolumn{6}{l}{\emph{Signif. Codes: ***: 0.01, **: 0.05, *: 0.1}}\\
      \end{tabular}
   \end{threeparttable}
\end{table}




\begin{table}[htbp]
   \centering
   \tiny
   \begin{threeparttable}[b]
      \caption{\label{tab:reg_has_inflows_entropy_tag_spend_sz_inc_var_quint} Effect of entropy on P(has savings) by income variability quintile}
      \begin{tabular}{lccccc}
         \tabularnewline \midrule \midrule
         Model:                       & (1)              & (2)              & (3)              & (4)              & (5)\\  
         Income variability quintile: & 1                & 2                & 3                & 4                & 5 \\   
         \midrule
         \emph{Variables}\\
         Entropy (48 cats, smooth)    & -0.019$^{***}$   & -0.020$^{***}$   & -0.017$^{***}$   & -0.022$^{***}$   & -0.024$^{***}$\\   
                                      & [-0.024; -0.015] & [-0.024; -0.016] & [-0.021; -0.013] & [-0.026; -0.018] & [-0.028; -0.019]\\   
         Month spend                  & 0.007$^{***}$    & 0.007$^{***}$    & 0.006$^{***}$    & 0.006$^{***}$    & 0.006$^{***}$\\   
                                      & [0.006; 0.008]   & [0.006; 0.008]   & [0.005; 0.008]   & [0.005; 0.007]   & [0.005; 0.007]\\   
         Month income                 & 0.014$^{***}$    & 0.010$^{***}$    & 0.011$^{***}$    & 0.010$^{***}$    & 0.010$^{***}$\\   
                                      & [0.011; 0.017]   & [0.008; 0.012]   & [0.009; 0.013]   & [0.009; 0.012]   & [0.009; 0.011]\\   
         Has income in month          & 0.093$^{***}$    & 0.075$^{***}$    & 0.064$^{***}$    & 0.056$^{***}$    & 0.066$^{***}$\\   
                                      & [0.074; 0.112]   & [0.055; 0.094]   & [0.047; 0.081]   & [0.040; 0.071]   & [0.052; 0.081]\\   
         Income variability           & -0.001           & 0.005$^{**}$     & -0.000           & 0.000            & -0.004$^{**}$\\   
                                      & [-0.003; 0.001]  & [0.001; 0.010]   & [-0.004; 0.003]  & [-0.001; 0.002]  & [-0.006; -0.001]\\   
         \midrule
         \emph{Fixed-effects}\\
         User                         & Yes              & Yes              & Yes              & Yes              & Yes\\  
         Year-month                   & Yes              & Yes              & Yes              & Yes              & Yes\\  
         \midrule
         \emph{Fit statistics}\\
         Observations                 & 223,462          & 215,151          & 208,417          & 201,319          & 195,378\\  
         R$^2$                        & 0.61181          & 0.59979          & 0.59070          & 0.59394          & 0.63550\\  
         Within R$^2$                 & 0.00597          & 0.00466          & 0.00502          & 0.00614          & 0.00864\\  
         \midrule \midrule
         \multicolumn{6}{l}{\emph{Clustered (User) co-variance matrix, 95\% confidence intervals in brackets}}\\
         \multicolumn{6}{l}{\emph{Signif. Codes: ***: 0.01, **: 0.05, *: 0.1}}\\
      \end{tabular}
   \end{threeparttable}
\end{table}








\subsection{Why does smoothing flip the direction of the effect}%
\label{sub:why_does_smoothing_flip_the_direction_of_the_effect}

As discussed in Section~\ref{sub:spending_profiles}, we create smooth entropy
measures by applying additive smoothing to the probability that a spending
transaction takes place in a given spending category by adding 1 to the
transaction count of that spending category in the numerator, and adding the
number of spending categories to the total number of spending transactions in
the denominator.


