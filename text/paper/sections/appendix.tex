% !TEX root = ../entropy.tex

\section{Data}%
\label{sec:data}


Data issues:

\citet{bourquin2020effects} argue that because some of the accounts in the data
will be joint accounts, units of observations should be tought of as
"households" rather than "users". We do not agree that this is the most prudent
approach. The validity of thinking of units as households depends on the
proportion of users in the data who add joint accounts and on the proportion of
transactions -- out of a user's total number of transactions -- additionally observed as a
result. Given that the sample is skewed towards younger individuals we think it
is unlikely that a majority of them has added joint accounts. Furthermore, it
seems reasonable to assume that in most cases, joint accounts are mainly used
for common household expenditures similar that are similar to those of a single
user (albeit in higher amounts), and are thus unlikely to alter the observed
spending profile much. Thus, we think of units of observations as individuals,
not households. 

Some accounts might be business accounts. Using versions of the algorightms
used by \citet{bourquin2020effects} to identify such accounts showed, however,
that such accounts only make up a tiny percentage of overall accounts and would
not influence our results. We thus do not exclude them.


Preprocessing steps
(provide detailes and links to relevant code files)
\begin{itemize}
    \item Duplicates handling.
    \item We trim all variables at the 1-percent level on the upper end of the
        distribution for variables that take non-negative values only and on
        both ends of the distribution for all other variables. We trim
        (replace outliers with missing values) rather than winsorise (replace
        outliers with the cutoff percentile value) because we believe that
        outliers result from errors in the data rather than represent genuine
        information.
    \item Actually, we don't do either of the above. With the harsher selection
        methods, the statistics are very reasonable, which, if anything, would
        suggest using winsorizing. However,
        [this](https://blogs.sas.com/content/iml/2017/02/08/winsorization-good-bad-and-ugly.html)
        article convincingly argues that we shouldn't do that in our case.
\end{itemize}
