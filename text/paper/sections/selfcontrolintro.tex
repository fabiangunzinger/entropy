% !TEX root = ../entropy.tex

\section{Introduction}%
\label{sec:introduction}

This paper investigates the degree to which self-control problems determine
individuals' levels of long- and short-term savings and discretionary
spending.

We measure self-control based on variation in spending on ``highly
discretionary spending'' that are tend to be goods that people either want to
reduce (fast food, gambling) or want to increase and maintain (gym membership,
health food). We use the within-person variation to measure self-control to
account for differences in preferences, following \citet{cherchye2017new}.

We hypothesise that lower levels of self-control are associated with higher
levels of discretionary spend and lower levels of savings.

Contribution:
\begin{itemize}
    \item We document within- and between-person variation in discretionary spending and
        savings behaviour.
        \begin{itemize}
            \item Compare within and between variation as in
                \citet{bogomolov2014daily} and table similar to Tab A.3 in
                \citet{cherchye2017new}.

            \item Document how self-control differs by age, gender, and
                income-levels.
        \end{itemize}

    \item We show that people with higher self-control issues spend more on
        highly discretionary items and have lower short- and long-term savings.

\end{itemize}

Limitations:
\begin{itemize}
    \item Regressing (discretionary spending in pounds) on (sd discretionary spending -
        based on value or number of txn) suffers from reverse causality issue
        if sd is a function of amount of discretionary spend. It's not obvious
        that it is, though.

    \item People have heterogeneous preferences over discretionary spend.
        Focusing on within-person fluctuations avoids confounding self-control
        issues with a difference in preferences.

    \item Some variation in discretionary spend may reflect a rational response
        to changes in prices and consumption budgets. Because we don't see the
        precise products that people buy and don't haven access to prices, we
        cannot take price changes into account. We can infer income and thus
        approximate budgets. Also, work by \citet{cherchye2017new}, who focus on food
        purchases, suggests that price responses or changes in consumption
        budgets account for about 20 percent of the variation in spending,
        which suggests an upper bound for such effects.

\end{itemize}


Ideas:
\begin{itemize}

    \item Check whether spend on highly discretionary spend tends to be
        ``reset'' around ``fresh-start'' points (New Year, Easter, day or
        month) by presenting figures like Fig. 2.3 in \citet{cherchye2017new}.
        This would really indicate that consumption is driven by self-control
        problem.

    \item Check that variation is not driven by seasonality.

\end{itemize}


Literature:
\begin{itemize}
    \item Relationship between self-control and poverty / low-income:
        \citet{mani2013poverty, haushofer2014psychology, bernheim2015poverty}

    \item Relationship between self-control and age:
        \citet{ameriks2007measuring, bucciol2012measuring}
\end{itemize}
