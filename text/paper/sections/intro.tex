% !TEX root = ../entropy.tex

\section{Introduction}%
\label{sec:introduction}

This paper documents variation in spending profiles of a large set of users of
a financial management app and shows that spending profiles are related to how
much users save and spend.

We can think of total savings as the sum of long-term and short-term savings.
Long-term savings are savings for retirement, either individually or through an
employer-linked pension scheme. These kinds of savings, and especially savings
through pension schemes, are well researched. In contrast, there is almost no
research on short-term savings, which comprise savings for particular goals
such as a new car, a holiday, or a wedding, and emergency savings to have a
buffer for unexpected events. (See nest2021supporting for more on emergency
savings) We aim to close this gap. In our data, we cannot distinguish between
goal-oriented and emergency savings, so we focus on total short-term
savings\footnote{MDB allows users to create custom tags and some users use them
to indicate the intended use for their savings transactions (e.g. "wedding",
"holidays"). But only a very small number of transactions have such tags, and
we do not pursue this further.}

Studying the determinants of``emergency savings'' is important because even
though around a quarter of adults in the UK and the US are unable to cover
irregular expenses like car and medical bills,\footnote{In the UK, 25 percent
    of adults would be unable to cover an unexpected bill of \pounds300
\citep{philipps2021supporting}, while in the US, about 30 percent would be
unable to cover a \$400 bill \citep{fed2022economic}.} there is little research
that studies why people have low levels of ``emergency savings'' and test ways
to help them build such savings.\footnote{In contrast, there is a large body of
    research on pension savigns.  Well-documented behavioural biases that help
    explain undersaving are, among others, present bias
    \citep{laibson1997golden, laibson2019intertemporal}, inertia
    \citep{madrian2001power}, over-extrapolation \citep{choi2009reinforcement},
    and limited self-control and willpower \citep{thaler1981economic,
    benhabib2005modeling, fudenberg2006dual, loewenstein2004animal,
gul2001temptation}. One danger of viewing low savings mainly as a result of
behavioural biases is that while these biases likely do play some role and
designing environments and tools to help correct them are thus part of the
solution, it is at least conceivable that this is an area where the focus on
behaviour-level solutions distracts from an effort to find more effective
society-level solutions, a danger inherent in behavioural science research
convincingly highlighted in \citet{chater2022frame}: if the main problem is
that many people are unable to earn enough to save, then the effectiveness of
helping them manage their low incomes more effectively pales in comparison with
efforts to help them earn more.}

Studying spending profiles are of interest because:
\begin{itemize}

    \item Our understanding of how people spend their money is based on survey
        data.

    \item Large-scale transaction-level data offers the possibility to study
        spending behaviour based on real-time data that are automatically
        collected for a large number of users. Such data has only become
        available ver recently and have not, thus far, been used to investigate
        systematically how people spend their money.

    \item Research in psychology suggests that disorder is maladaptive and
        associated with a range of negative outcomes such as impaired executive
        function \citep{vernon2016predictors}, lower cognitive inhibition
        \citep{mittal2015cognitive}, and activation of anxiety-related neural
        circuits \citep{hirsh2012psychological}.

    \item In the study of human behaviour, more chaotic behaviour has been
        found to predict the a higher number of visits to and higher spend in
        supermarkets \citep{guidotti2015behavioral}, higher calorie intake
        \citep{skatova2019those} and financial distress
        \citep{muggleton2020evidence}.

\end{itemize}


Contribution:
\begin{itemize}

    \item Systematically documenting ``emergency savings'' patterns.

    \item Systematically documenting variation in spending profiles.

    \item Showing that within-user irregularity in spending profiles is
        associated with lower ``emergency savings'' and higher ``discretionary
        spending''.

\end{itemize}


Literature:
\begin{itemize}

    \item Main 1: Understanding emergency savings behaivour (nest, aspen
        reports), \citep{sabat2019rules} for sources on short-term savings
        literature, \citet{colby2013savings} for lit on savings goals.  See
        \citet{colby2013savings} has useful literature review on short-term
        savings and suggests that subgoals can increase willingness to forego
        short-amounts in the present because they move the reference point in a
        prospect-theory framework.


    \item Main 2: Understanding effect of behavioural entropy - eliciting
        useful personality characteristics from large-scale data

    \item Also 2: Use of high-frequency transaction data (itself a
        sub-literature of use of newly available large-scale datasets)

\end{itemize}


Structure of paper:


\begin{itemize}

    \item Part I: describtive states of how and when people spend and save

    \item Part II: define measures that characterise spend profile

    \item Part III: regression analysis

\end{itemize}


Literature:
\begin{itemize}
    \item Savings lit: \citet{lunt1991psychological, oaten2007improvements, }
\end{itemize}
% \citet{muggleton2020evidence} find that consumption entropy over categories
% correlates with financial distress.

% \citet{davenport2020spending} study the impact of COVID-19 on the spending and
% savings behaviour of MDB users.

% \citet{becker2017does} finds that access to a fintech money management app
% increases first-time savings and savings account balances among 65,000 customers
% of a large European bank but that update is negatively correlated with financial
% sophistication.

% We use the following nomenclature throughout:
% \begin{description}
%     \item[user] individual
%     \item[tag] spending category
% \end{description}


