% !TEX root = ../entropy.tex


\begin{itemize}

    % what we do

    \item This paper explores the effect of spending profiles on financial
        outcomes.

    \item Spend profile: distribution of txn spend txns across txn categories.

    \item Financial outcomes: spending, saving, overdraft fees

    % why it's important

    \item Financial wellbeing is important.

    \item People in UK and US also don't have enough to cover unexpected
        outlays. (See reports)

    \item This has important consequences:

        \begin{itemize}

            \item Short-term: financial well-being (see reports)

            \item Long-term (viscious cycle): scarcity hypothesis - makes it
                harder to focus on important things (plan for retirement, focus
                on healthy lifestyle, support children, ...) and might lead to
                vicious cycle (less savings leading to increased risk of
                financial hardship leading to more stress leading to less
                savings...)

        \end{itemize}

    \item Spending and savings behaviour is important component - it's not all
        about lack of income.

    \item Structure paper:

        \begin{itemize}

            \item Part I: describtive states of when people spend, save, and pay od fees

            \item Part II: define measures that characterise spend profile

            \item Part III: regression analysis

            \item Part IV: cluster analysis: do outcomes differ by groups? (poss sep ml
                focused paper)

            \item Part V: predict non-saving, high-spend, and od fee months (poss in
                separate paper with above section)

        \end{itemize}

    % contribution

    \item While there is large literature on long-term / pension savings, very
        little work on short-term financial spending and savings behaviour.

    \item In doing so, we aim to contribute to three literatures:

    \begin{itemize}

        % update once we have results 
        \item Main 1: Understanding emergency savings behaivour (nest, aspen reports)

        \item Main 2: Understanding effect of behavioural entropy - eliciting
            useful personality characteristics from large-scale data

        \item Also 1: More broadly: part of savings literature (pension literature,
            savings buffer)

        \item Also 2: Use of high-frequency transaction data (itself a sub-literature of
            use of newly available large-scale datasets)
    \end{itemize}
    \item
\end{itemize}



% Literature:

% \citet{sabat2019rules} literature discussion on how much people should save. 

% \citet{muggleton2020evidence} find that consumption entropy over categories
% correlates with financial distress.

% \citet{davenport2020spending} study the impact of COVID-19 on the spending and
% savings behaviour of MDB users.

% \citet{baker2021household} summarises literature that uses mass financial
% transaction data to study household financial behaviour.

% \citet{becker2017does} finds that access to a fintech money management app
% increases first-time savings and savings account balances among 65,000 customers
% of a large European bank but that update is negatively correlated with financial
% sophistication.

% \citet{colby2013savings} has useful literature review on short-term savings and
% suggests that subgoals can increase willingness to forego short-amounts in the
% present because they move the reference point in a prospect-theory framework.

% We use the following nomenclature throughout:
% \begin{description}
%     \item[user] individual
%     \item[tag] spending category
% \end{description}


