% !TEX root = ../entropy.tex

\section{Introduction}%
\label{sec:introduction}

Story:
\begin{itemize}

    \item This paper tests whether people save less in times when their lives
        are more chaotic. We define savings as ... and capture chaotic lives
        using ...

    \item Many people in UK and US struggle to save
        enough.

    \item This is well documented and thoroughly studied for retirement
        savings.

    \item But these are not only savings. People in UK and US also don't have
        enough to cover unexpected outlays.

    \item This could have important consequences: scarcity hypothesis - makes
        it harder to focus on important things (plan for retirement, focus on
        healthy lifestyle, support children, ...) and might lead to vicious
        cycle (scarcity research)

    \item This paper aims to start to fill gap and study short-term savings
        behaviour.

    \item In doing so, we aim to contribute to three literatures:

    \begin{itemize}
        \item Understanding savings behaivour

        \item Understanding effect of entropy (muggleton)

        \item Use of high-frequency transaction data (itself a sub-literature of
            use of newly available large-scale datasets)
    \end{itemize}
\end{itemize}


% Literature:

% \citet{muggleton2020evidence} find that consumption entropy over categories
% correlates with financial distress.

% \citet{davenport2020spending} study the impact of COVID-19 on the spending and
% savings behaviour of MDB users.

% \citet{baker2021household} summarises literature that uses mass financial
% transaction data to study household financial behaviour.

% \citet{becker2017does} finds that access to a fintech money management app
% increases first-time savings and savings account balances among 65,000 customers
% of a large European bank but that update is negatively correlated with financial
% sophistication.

% \citet{colby2013savings} has useful literature review on short-term savings and
% suggests that subgoals can increase willingness to forego short-amounts in the
% present because they move the reference point in a prospect-theory framework.

% We use the following nomenclature throughout:
% \begin{description}
%     \item[user] individual
%     \item[tag] spending category
% \end{description}


