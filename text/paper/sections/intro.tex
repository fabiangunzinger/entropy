% !TEX root = ../entropy.tex

\section{Introduction}%
\label{sec:introduction}

This paper documents variations in spending profiles and payments into
emergency savings for a large set of users of a financial management app
and shows that spending profiles predict emergency savings.

We define emergency savings as inflows into savings accounts. These savings
will be made up of savings for particular goals -- a new car, a holiday, a
wedding -- and savings directed towards building up a buffer for financial
emergencies. Because in our data we cannot distinguish between these two cases,
we refer to all of them as emergency savings.\footnote{MDB allows users to
    create custom tags and some users use them to indicate the intended use for
    their savings transactions (e.g. ``wedding'', ``holidays''). But only a very
small number of transactions have such tags, and we do not pursue this
further.} These short-term savings are distinct from long-term savings aimed to
build up funds for retirement, either through individually owned pension and
investment vehicles or employer-linked pension schemes. Both kinds of saving
are important for financial well-being, yet while there is a large literature
on pension savings, little is known about how people save for the
short-term.\footnote{Well-documented behavioural biases that help explain
    undersaving for pensions are, among others, present bias
    \citep{laibson1997golden, laibson2019intertemporal}, inertia
    \citep{madrian2001power}, over-extrapolation \citep{choi2009reinforcement},
    and limited self-control and willpower \citep{thaler1981economic,
    benhabib2005modeling, fudenberg2006dual, loewenstein2004animal,
gul2001temptation}. One danger of viewing low savings mainly as a result of
behavioural biases is that while these biases likely do play some role and
designing environments and tools to help correct them are thus part of the
solution, it is at least conceivable that this is an area where the focus on
behaviour-level solutions distracts from an effort to find more effective
society-level solutions, a danger inherent in behavioural science research
convincingly highlighted in \citet{chater2022frame}: if the main problem is
that many people are unable to earn enough to save, then the effectiveness of
helping them manage their low incomes more effectively pales in comparison with
efforts to help them earn more.}

Studying the determinants of emergency savings is important because around a
quarter of adults in the UK and the US are unable to cover irregular expenses
like car and medical bills: In the UK, 25 percent of adults would be unable to
cover an unexpected bill of \pounds300 \citep{philipps2021supporting}, while in
the US, about 30 percent would be unable to cover a \$400 bill
\citep{fed2022economic}. Similarly, resarch in the UK has shown that having
\pounds1000 in savings reduces by more than half a household's chances of
falling into debt that leads to financial problems
\citep{philipps2021supporting}.


Studying spending profiles is of interest because:
\begin{itemize}

    \item Our understanding of how people spend their money is based on survey
        data.

    \item Large-scale transaction-level data offers the possibility to study
        spending behaviour based on real-time data that are automatically
        collected for a large number of users. Such data has only become
        available ver recently and have not, thus far, been used to investigate
        systematically how people spend their money.

    \item Research in psychology suggests that disorder is maladaptive and
        associated with a range of negative outcomes such as impaired executive
        function \citep{vernon2016predictors}, lower cognitive inhibition
        \citep{mittal2015cognitive}, and activation of anxiety-related neural
        circuits \citep{hirsh2012psychological}. In the study of human
        behaviour, more chaotic behaviour has been found to predict the a
        higher number of visits to and higher spend in supermarkets
        \citep{guidotti2015behavioral}, higher calorie intake
        \citep{skatova2019those} and financial distress
        \citep{muggleton2020evidence}.

\end{itemize}


We hypothesise that less predictable spending patterns are associated with a
lower probability for making payments into emergency savings accounts. Possible
channels:
\begin{itemize}
    \item Disorder (personal life or environment): leads to more impulsive
        shopping behaviour and makes forgetting to save more likely.

    \item Scarcity: life challenges focus attention away from deliberate
        shopping, causing hore impulse purchases, and make fortetting to save
        more likely.
\end{itemize}



What we do: 
\begin{itemize}

    \item Systematically documenting emergency savings patterns.

    \item Systematically documenting variation in spending profiles.

    \item Showing that unpredictability in spending profiles is
        associated with lower emergency savings.

\end{itemize}


Contribution to literatures:
\begin{itemize}

    \item Understanding emergency savings behaivour (nest, aspen
        reports), \citep{sabat2019rules} for sources on short-term savings
        literature, \citet{colby2013savings} for lit on savings goals.  See
        \citet{colby2013savings} has useful literature review on short-term
        savings and suggests that subgoals can increase willingness to forego
        short-amounts in the present because they move the reference point in a
        prospect-theory framework. \citet{philipps2021supporting} present
        results from an employer-linked initiative that offers employees to
        have a portion of their salary automatically transferred into a savings
        pot. Policy literature: \citep{can2019improving,cfpb2017financial,
        mps2018building}. Older literature:Savings lit:
        \citet{lunt1991psychological, oaten2007improvements}

    \item Understanding effect of behavioural entropy - eliciting
        useful personality characteristics from large-scale data.

    \item Use of high-frequency transaction data (itself a
        sub-literature of use of newly available large-scale datasets).

\end{itemize}

