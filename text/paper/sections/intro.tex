% !TEX root = ../entropy.tex


\begin{itemize}

    %what are we doing?
    \item This paper aims to better understand the drivers of emergency
        savings. 

        \begin{itemize}
            \item We test the important of commonly discussed determinants
                (financial behaviour, planning, characteristics), see reports.

            \item Entropy has recently been found to affect financial outcomes.
                We test whether entropy has impact above and beyond traditional
                factors, or can be thought of as a summary statistic of (some)
                of them.
        \end{itemize}



    % why does it matter?

    \item Large literature on savings, but academic literature almost
        exclusively focused on pensions.

    \item Short-term savings matter, too. 

    \item People in UK and US also don't have enough to cover unexpected
        outlays. (See reports)

    \item This has important consequences:

        \begin{itemize}

            \item Short-term: financial well-being (see reports)

            \item Long-term (viscious cycle): scarcity hypothesis - makes it
                harder to focus on important things (plan for retirement, focus
                on healthy lifestyle, support children, ...) and might lead to
                vicious cycle (less savings leading to increased risk of
                financial hardship leading to more stress leading to less
                savings...)

        \end{itemize}

    \item This paper aims to start to fill gap and study short-term savings.



    % main findings
    \item Main findings:

        \begin{itemize}
            \item Chaotic lifestyle impairs saving above and beyond total
                amount spend and income earned, implying that lack of saving is
                not only due to incomes too low to save.

            \item What is entropy? Convenient measure to pick up stress in
                person's life?

            \item Mechanism: scarcity.

        \end{itemize}

    % contribution
    \item In doing so, we aim to contribute to three literatures:

    \begin{itemize}
        \item Main 1: Understanding emergency savings behaivour (nest, aspen reports)

        \item Main 2: Understanding effect of behavioural entropy - eliciting
            useful personality characteristics from large-scale data

        \item Also 1: More broadly: part of savings literature (pension literature,
            savings buffer)

        \item Also 2: Use of high-frequency transaction data (itself a sub-literature of
            use of newly available large-scale datasets)
    \end{itemize}
\end{itemize}

% Literature:

% \citet{sabat2019rules} literature discussion on how much people should save. 

% \citet{muggleton2020evidence} find that consumption entropy over categories
% correlates with financial distress.

% \citet{davenport2020spending} study the impact of COVID-19 on the spending and
% savings behaviour of MDB users.

% \citet{baker2021household} summarises literature that uses mass financial
% transaction data to study household financial behaviour.

% \citet{becker2017does} finds that access to a fintech money management app
% increases first-time savings and savings account balances among 65,000 customers
% of a large European bank but that update is negatively correlated with financial
% sophistication.

% \citet{colby2013savings} has useful literature review on short-term savings and
% suggests that subgoals can increase willingness to forego short-amounts in the
% present because they move the reference point in a prospect-theory framework.

% We use the following nomenclature throughout:
% \begin{description}
%     \item[user] individual
%     \item[tag] spending category
% \end{description}


