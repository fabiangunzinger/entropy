% !TEX root = ../entropy.tex

\section{Introduction}%
\label{sec:introduction}

This paper documents variation in spending profiles -- the way spending
transactions are allocated across product of merchant categories -- for a large
set of users of a financial management app and shows that spending profiles
predict ``emergency savings'' -- short-term savings intended as a buffer
against unexpected financial shocks.

Emergency savings are distinct from long-term savings aimed to build up funds
for retirement, either through individually owned pension and investment
vehicles or employer-linked pension schemes. Both kinds of saving are important
for financial wellbeing \citep{mps2018building, cfpb2017financial}, yet while
there is a large literature on pension savings -- particularly on the effect on
auto-enrollment on savings plan participation -- little is known about how
individuals save for the short-term.\footnote{For main contributions to the to
literature on auto-enrollment into savings plans, see \citet{madrian2001power,
choi2002defined, choi2004better, beshears2009importance}.} Apart from
\citet{beshears2020building}, who study design options for automated
contributions to emergency saving pots, and \citet{phillips2021supporting}, who
test such a design in the UK, there is no research on how we can help
individuals build such emergency savings. Furthermore, around a quarter of
adults in the UK and the US are unable to cover irregular expenses like car and
medical bills: in the UK, 25 percent of adults would be unable to cover an
unexpected bill of \pounds300 \citep{phillips2021supporting}, while in the US,
about 30 percent would be unable to cover a \$400 bill \citep{fed2022economic}.
Research from the UK suggests, however, that even having \pounds1000 in savings
reduces by more than half a household's chances of falling into debt that leads
to financial problems \citep{phillips2021supporting}.

There is a large literature on the determinants of financial behaviour, and
some of the main factors that have been identified to influence have
individuals and households make financial decisions are: cognitive limitations
and financial literacy \citep{agarwal2009age, agarwal2013cognitive,
    korniotis2011older, agarwal2010learning, fernandes2014financial,
    jorring2020financial}; time-preferences and self-control
    \citep{frederick2002time, read2018intertemporal, ericson2019intertemporal,
    cohen2020measuring}; attitude towards money and spending
    \citep{rick2008tightwads, rick2011fatal}; ones perceived locus of control
    \citep{perry2005control}, degree of optimism \citep{puri2007optimism},
    ability to frame decisions broadly rather than narrowly
    \citep{kumar2008decision}, and propensity to gamble
    \citep{kumar2009gambles}; ones social network \citep{bailey2018economic,
        kuchler2021social}; the degree of ones financial planning
        \citep{ameriks2003wealth}; and habits \citep{blumenstock2018defaults,
            schaner2018persistent, de2013deposit}.\footnote{For two thorough
        reviews, see \citet{agarwal2017shapes} and
    \citet{greenberg2019financial}.}

In this paper, we use simple summary statistics of individuals' spending
profiles and test whether differences in spending profiles are predictive of
the frequency with which individuals make transactions into their savings
accounts. To summarise spending profiles, we calculate the entropy of an
individual's spending profile in a given period, which captures the
predictability of spending transactions in said period. We focus on spending
profiles for two reasons: first, our understanding of how individuals spend
their money is largely based on survey data from a relatively small number of
individuals. Large-scale transaction data of the type used in this paper has
become available to researchers only in recent years and has not, to the best
of our knowledge, been used to explore patterns in consumer spending behaviour. 

Second, research suggests that spending profiles might reflect circumstances in
an individual's life that are also related to their saving behaviour. For
instance, \citet{muggleton2020evidence} find that higher spending entropy
predicts financial distress, and hypothesise that this is because disorder in
ones personal life might simultaneously be reflected in ones spending behaviour
and make it harder to deal with financial obligations, based on a literature in
psychology that suggests that disorder is associated with a range of negative
outcomes such as impaired executive function \citep{vernon2016predictors},
lower cognitive inhibition \citep{mittal2015cognitive}, and activation of
anxiety-related neural circuits \citep{hirsh2012psychological}. In addition to
financial outcomes, high entropy behaviour has also been found to be associated
with higher calorie intake \citep{skatova2019those} and with a higher number of
visits to and higher spend in supermarkets \citep{guidotti2015behavioral}.

Our results show that spending entropy is predictive of the frequency of
savings transactions, with an effect size similar to that of an increase in
monthly income income of between \pounds1,000 and \pounds2,000. This holds true
if we control for simple component parts of entropy, suggesting that the
non-linear way in which entropy combines these components captures something
about spending profiles that is of relevance. The results are also largely
consistent across entropy measures calculated based on different product and
merchant categories. However, they direction of the effect changes for
different types of entropy that differ in the way in which we treat product or
merchant categories in which an individual makes no transactions in a given
period. Exploratory research suggests that the number of such zero count
categories, together with the variation of counts for categories with a
positive number of transactions, goes some way in explaining the outcome. But
more work is necessary to more fully understand this outcome, which we leave
for future research.

Apart from the small literature on emergency savings mentioned above, our work
contributes to a broader literature that uses financial-transaction data
from banks or financial aggregator apps to understand consumer financial
behaviour. For instance, \citet{kuchler2020sticking} use data from a
financial aggregator app to estimate time preferences. Similar data has been
used to show that consumer spending varies across the pay cycle
\citep{gelman2014harnessing,olafsson2018liquid}, to test the consumer spending
response to exogenous shocks \citep{baker2018debt,baugh2014disentangling}, and
to better understand the generational differences in financial platform usage
patterns \citep{carlin2019generational}. Some researchers use transaction-data
directly provided by banks. \citet{ganong2019consumer} show that consumer
spending drops sharply after the predictable income drop from exhausting
unemployment insurance benefits, \citet{meyer2018fully} analyse how individuals
reinvest realised capital gains and losses, and \citet{muggleton2020evidence}
show that chaotic spending behaviour is a harbinger of financial
distress.\footnote{For a comprehensive review of the literature using financial
transaction data, see \citet{baker2022household}.}

The remainder of this paper is organised as follows: Section~\ref{sec:methods}
discusses data pre-processing and our methodological approach,
Section~\ref{sec:results} presents the main results and further exploratory
analysis, and Section~\ref{sec:conclusion} concludes.

