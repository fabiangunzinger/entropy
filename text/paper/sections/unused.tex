
\section{Effect of entropy on overdraft fees}%
\label{sub:effect_of_entropy_on_overdraft_fees}

\begin{itemize}

    \item Table~\ref{tab:reg_entropy_odfees} replicates results from
        \citet{muggleton2020evidence} Tables S20 (Columns 1 and 2) and Table
        S40 (columns 3 and 4).

    \item Similar to their results, higher entropy is positively related to
        negative financial outcomes.

    \item Yet, contrary to their findings, the effect becomes stronger in the
        presence of fixed effects.

    \item Also, in contrast to above findings, using non-smoothed entropy
        doesn't reverse the effect but strengthen it.

    \item Differences to \citet{muggleton2020evidence} could be due to a number
        of factors: we use a different dataset, a different and self-selected sample of people
        that is skewed towards younger and wealthier individuals, as well as a
        longer panel.

\end{itemize}


\begin{table}[htbp]
   \centering
   \caption{\label{tab:reg_entropy_odfees} Effect of entropy on overdraft fees}
   \begin{scriptsize}
      \begin{tabular}{lcccc}
         \tabularnewline\midrule\midrule
         Dependent Variable: & \multicolumn{4}{c}{has\_od\_fees}\\
         Model:                      & (1)                     & (2)                     & (3)            & (4)\\
         \midrule \emph{Variables} &   &   &   &  \\
         Entropy (tag-based, smooth) & 0.0040$^{**}$           &                         & 0.0052$^{*}$   &   \\
                                     & (0.0018)                &                         & (0.0027)       &   \\
         Entropy (tag-based)         &                         & 0.0745$^{***}$          &                & 0.0318$^{***}$\\
                                     &                         & (0.0029)                &                & (0.0044)\\
         Spend communication         & 0.6974$^{***}$          & 0.5952$^{***}$          & 0.1294$^{***}$ & 0.1025$^{***}$\\
                                     & (0.0256)                & (0.0258)                & (0.0385)       & (0.0384)\\
         Spend finance               & 0.0256$^{***}$          & 0.0192$^{***}$          & 0.0124$^{***}$ & 0.0110$^{***}$\\
                                     & (0.0028)                & (0.0028)                & (0.0037)       & (0.0037)\\
         Spend hobbies               & 0.0040                  & -0.1075$^{***}$         & 0.0411         & 0.0106\\
                                     & (0.0283)                & (0.0284)                & (0.0263)       & (0.0265)\\
         Spend household             & -0.0011                 & 0.0018                  & 0.0027         & 0.0033\\
                                     & (0.0016)                & (0.0016)                & (0.0021)       & (0.0021)\\
         Spend other                 & 0.0188$^{***}$          & 0.0072$^{*}$            & -0.0053        & -0.0091$^{**}$\\
                                     & (0.0040)                & (0.0040)                & (0.0040)       & (0.0040)\\
         Spend motor                 & 0.1373$^{***}$          & 0.0441$^{***}$          & 0.0326         & 0.0068\\
                                     & (0.0154)                & (0.0157)                & (0.0208)       & (0.0208)\\
         Spend retail                & 0.0109                  & -0.0124$^{*}$           & 0.0018         & -0.0063\\
                                     & (0.0073)                & (0.0074)                & (0.0068)       & (0.0070)\\
         Spend services              & -0.0073$^{**}$          & 0.0085$^{**}$           & -0.0054        & -0.0026\\
                                     & (0.0036)                & (0.0035)                & (0.0039)       & (0.0038)\\
         Spend travel                & -0.0329$^{***}$         & -0.0449$^{***}$         & -0.0041        & -0.0078$^{**}$\\
                                     & (0.0049)                & (0.0049)                & (0.0036)       & (0.0036)\\
         Female                      & 0.0429$^{***}$          & 0.0367$^{***}$          & -0.3720        & -0.3891\\
                                     & (0.0030)                & (0.0030)                & (47,268.6)     & (47,267.8)\\
         Age                         & -0.0026$^{***}$         & -0.0026$^{***}$         &                &   \\
                                     & (0.0001)                & (0.0001)                &                &   \\
         Year income                 & -0.0015$^{***}$         & -0.0015$^{***}$         & 0.0005$^{*}$   & 0.0005\\
                                     & ($9.26\times 10^{-5}$) & ($9.22\times 10^{-5}$) & (0.0003)       & (0.0003)\\
         (Intercept)                 & 0.2793$^{***}$          & 0.2746$^{***}$          &                &   \\
                                     & (0.0057)                & (0.0056)                &                &   \\
         \midrule \emph{Fixed-effects} &   &   &   &  \\
         User id                     &                         &                         & Yes            & Yes\\
         Calendar month              &                         &                         & Yes            & Yes\\
         \midrule \emph{Fit statistics} &   &   &   &  \\
         Observations                & 82,589                  & 82,589                  & 82,589         & 82,589\\
         R$^2$                       & 0.02522                 & 0.03275                 & 0.63492        & 0.63561\\
         Within R$^2$                &                         &                         & 0.00162        & 0.00352\\
         \midrule\midrule\multicolumn{5}{l}{\emph{Signif. Codes: ***: 0.01, **: 0.05, *: 0.1}}\\
      \end{tabular}
   \end{scriptsize}
\end{table}





\paragraph{Shopping-time based entropy}
\label{par:shopping_time_based_entropy}

We calculate entropy based on the probability of $(day of week, merchant)$
tuples, where we follow \citet{guidotti2015behavioral} and bin \textit{day of
week} into \textit{weekends} and \textit{weekday}, to reduce excessive
fluctuations. Because banks tend to process weekend transactions on Monday, as
shown in Figure~\ref{fig:spending}, we cannot distinguish transactions made on
Saturdays or Sundays from those made on Mondays, and thus classify all of them
as weekend transactions. We drop the about 25 percent of transactions for which
we cannot identify a merchant. The alternative would be leaving these
transactions in the sample and treating ``unknown merchant'' as a single
merchant. But for user-months for which the merchant is unknown for all
transactions, this would lead to an entropy score of 0, which is undesireable.

\paragraph{Grocery shop entropy}%
\label{par:grocery_shop_entropy}

We consider purchases at Tesco, Sainsbury's, Asda, Morrisons, Aldi, Co-op,
Lidl, Waitrose, Iceland, and Ocado, which have a combined market share of 96.5
percent.

