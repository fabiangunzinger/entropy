% !TEX root = ../entropy.tex

\section{Conclusion}
\label{sec:conclusion}

In this paper, we calculate simple summary statistics for spending profiles for
a large number of users of a UK-based financial aggregator app and show that
the predictability of spending, captured using entropy, is predictive of the
frequency with which users make transactions into their savings accounts. The
effect of spending entropy on savings is statistically significant and
economically large, and robust across a number of different specifications and
entropy measures based on different product and merchant categories. The
direction of the effect depends, however, on the handling of categories in
which individuals make no transactions in a given month. While our results give
some indication as to how we can explain this result, further research is
needed.

As part of a fuller understanding of said sign reversal, more work generally is
needed on understanding to what extend entropy is useful in capturing
meaningful information about individuals' lives, and what particular ways of
calculating entropy are most appropriate and informative in different
circumstances. For instance, while, in this paper, we have calculated entropy
based on the number of transactions in a given category, there are a number of
alternatives. We could calculate profiles based on the distribution of
transaction values rather than counts. We could also calculate profiles based
on inter-temporal rather than intra-temporal distributions, focusing on
consistency of purchasing behaviour over time rather than on predictability at
any given time \citep{krumme2013predictability}. Further, we could focus on
time-based rather than category-based measures, focusing, for instance, on
whether purchases of the same type tend to occur on the same day of the week
\citep{guidotti2015behavioral}. Finally, one could also create composite
measures based on principal component analysis, an approach used in
\citet{eagle2010network}.

Finally, work like ours inherently risks portraying low savings mainly as a
result of behavioural biases, implying that if can help people overcome these
biases, they will be able to save more. However, as convincingly argued in
\citet{chater2022frame}, for problems where the main solution lies not in
individual behaviour change but public policies, such an approach runs the risk
of moving attention and resources away from required society-wide action.
Research by the Money and Pension Service in the UK \citep{mps2018building} and
the Consumer Financial Protection Bureau \citep{cfpb2017financial} suggests
that while low levels of savings are -- unsurprisingly -- often concentrated
among very low income individuals or households, individual behaviour still
plays a large role in explaining differences in savings levels across income
groups. Hence, addressing low levels of savings is a matter for both public
policy and individual behaviour change. But it is important to ensure that
focusing on the latter does not distract from the former.
